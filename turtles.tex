\documentclass{article}[letter,10pt]
\usepackage{biblatex}[encoding=utf8,backend=biber]
\addbibresource{turtles.bib}
\AtBeginBibliography{\setcounter{maxnames}{5}\setcounter{minnames}{5}}
\usepackage{geometry}[tmargin=1in,bmargin=1.5in,lmargin=1in,rmargin=1in]
\pagenumbering{gobble}

\title{Release the TURTLES\\
  \large{A Pure Tcl Interface to Dynamic Proc Tracing}}
\author{
  Michael Yantosca\\
  FlightAware \\
  michael.yantosca@flightaware.com
}

\begin{document}

\maketitle

\begin{abstract}
  Proper dynamic program analysis requires solid collection of call frequency,
  relationships, and timing. Typically, this is achieved with a suite of
  language-centric tools, e.g., valgrind (C/C++), the GHC profiling subsystem
  (Haskell). Tcl appears to lack similar tools or at least requires significant
  language extensions to meet this need. Consequently, this work introduces
  the Tcl Universal Recursive Trace Log Execution Scrutinizer (TURTLES), a
  simple proc tracing interface that yields easily aggregated and analyzed
  call records. By adhering to pure Tcl, the endeavor provides an enlightening
  étude in the pain points of developing such tooling in the language.
\end{abstract}

\section{Introduction}{
  \paragraph{}{
    The TURTLES project began as a small task meant to investigate the
    relationship between different procs within the Tcl repositories comprising
    the FlightAware codebase, particularly for the Multi-Machine HyperFeed (MMHF)
    project. The results of the analysis would be leveraged to inform
    refactoring efforts to improve the performance of the target program.
    Inspired by callgrind\autocite{callgrind} and the GHC profiling
    subsystem\autocite{ghcprof}, the TURTLES project aimed to capture not
    only the call-graph relationships but also performance metrics so that
    users of the library could consolidate correlated calls into logical
    modules as well as pinpoint execution hotspots.
  }
  \paragraph{}{
    Some initial investigation was made into extant facilities within
    the Tcl base or community packages that might achieve this purpose.
    Shannon Noe suggested cmdtrace\autocite{tcl::cmdtrace} and
    disassemble\autocite{tcl::disassemble} as potential candidates.
    After reviewing the documentation, the disassembler appeared to
    be worth investigating as a tool for static analysis in future work.
    To determine the suitability of cmdtrace for dynamic analysis, some dry
    runs were executed on toy programs with cmdtrace turned on. The resulting
    output was verbose and would require extra parsing to convert into
    a usable format. The parsing work was not necessarily prohibitive, but
    complex programs like MMHF could potentially generate unmanageable amounts
    of data, and so it was deemed that a more space-sensitive approach was
    required.  To this end, constraints were placed on the scope of the TURTLES
    project to only examine the immediate caller/callee relationship and provide
    execution timings inclusive of the total time spent both in the callee and
    in the callee's subcalls.
  }
  \paragraph{}{
    The TURTLES project also served as an introductory étude in Tcl introspection.
    As a neophyte in the language, working on the TURTLES project offered an
    accelerated course in namespace management and execution semantics. Pursuing
    a pure Tcl approach forced careful consideration of the costs incurred by
    the profiling overhead and exposed some pitfalls in Tcl that might
    be improved in future revisions of the language to make it more accessible
    and productive for both novice and experienced developers.
  }
  \paragraph{}{
    In order to maximize portability, one tacit goal of the TURTLES is to have
    minimal dependencies on other packages. The list of required packages is
    given as follows:
    \begin{itemize}
    \item{Tcl 8.5\autocite{tcl::85} or 8.6\autocite{tcl::86}}
    \item{Tclx\autocite{tcl::Tclx}}
    \item{Thread\autocite{tcl::Thread}}
    \item{cmdline\autocite{tcl::cmdline}}
    \item{platform\autocite{tcl::platform}}
    \item{sqlite3\autocite{tcl::sqlite3}}
    \item{tcltest\autocite{tcl::tcltest}}
    \end{itemize}
    Building the project requires GNU make or a compatible make utility.
    Code documentation can be generated by doxygen, but installation does
    not require it. The project repository README provides instructions
    on building and installing the TURTLES project, as well as instructions
    for use within a given user-defined program. Some aspects of usage
    will also be covered in the Design and Implementation sections of
    this report.
  }
  \paragraph{}{
    The remainder of this report is organized in the following manner.
    The Design section covers the abstract design decisions and data flow
    for the TURTLES project. The Implementation section expounds on the
    Design section with specific details about how the design was realized.
    The Experiments section exhibits the results of employing the TURTLES
    project on both toy examples and on the larger MMHF project.
    In the Conclusions section, the experimental results are analyzed
    to evaluate the performance and output of the TURTLES project.
    Finally, in the Future Work section, some avenues for ongoing research
    are considered in light of the aforementioned conclusions with the goal of
    improving the TURTLES project so as to encourage adoption and enhance
    the set of available tools for Tcl development.
  }
}

\section{Design}{
  A few guiding principles informed the design of the TURTLES project, namely
  minimization of overhead, correctness of collection, ease of use, and legibility
  of results. For simplicity, the call records do not attempt to store information
  for complete traces through the full call stack but rather capture the immediate
  caller-callee relationship.

  Starting and stopping data collection are each achieved by designated procs.

  \subsection{Initialization}{
    Starting data collection is achieved by a designated proc which sets up
    the necessary ephemeral and final storage for call records per arguments
    supplied by the user. A trace handler is added to the \texttt{proc} command on exit
    to bootstrap the assignment of trace handlers for entry and exit to every
    proc defined thereafter.

    For best results, it is recommended to make the initialization call as early
    as possible to capture the greatest number of proc definitions. Currently,
    the TURTLES project does not support retroactively adding proc handlers
    to procs already defined.

  }
  \subsection{Collection}{
    The entry and exit handlers capture the times of entry and exit into and out of
    the covered proc and record this in the form of a call point record. The caller
    and callee information in each call point record is normalized out to a collection
    of unique proc definition records.

    \paragraph{Proc Definition}{
      Each proc defined after data collection starts is stored in a proc definition
      record. This record is a triple of the proc name, a unique integral identifier,
      and the time of definition. The proc name is the unaliased, fully-qualified
      name of the proc, and the integral identifier is a hash of this value.
      The time of definition is canonically represented in UNIX epoch microseconds.
    }
    \paragraph{Call Point}{
      Each call made is recorded in the form of a call point record. This record
      is a quintuple consisting of a caller ID, callee ID, trace ID, time of
      entry, and time of exit. The caller and callee IDs correspond to unique
      integral identifiers in the set of proc definition records. The trace ID is a
      unique identifier for distinguishing separate calls with the same caller
      and callee and is calculated deterministically. Time of entry and exit are
      canonically represented in UNIX epoch microseconds.
    }
    \paragraph{Persistence}{
      Persistence of call records is handled either directly or in a staged fashion
      depending on the arguments supplied by the user. In direct mode, the call
      records are captured and stored immediately in the final storage. This has
      the benefit of retaining the most information in case execution is interrupted
      but at the cost of speed. In staged mode, the call records are captured and
      stored in ephemeral storage while a background task or thread periodically
      transfers unfinalized records in bulk to the final persistent storage. The
      immediate processing cost of each call record is reduced, but the risk of
      greater overall loss in cases of interruption is increased.
    }
  }
  \subsection{Finalization}{
    Ending data collection is achieved by a designated proc which disables
    all the relevant trace handlers along with the handler on the \texttt{proc}command.
    If collection was operating in staged mode, the ephemeral storage is flushed
    to the final storage.
  }
  \subsection{Analysis}{
    For post-hoc analysis, a clustering tool is provided to construct a call graph
    whose edges are the immediate caller-callee pairings and nodes are individual
    procs. The edge weights are defined as the number of invocations of the callee
    by the caller.

    In anticipation of a highly-connected graph with many nodes, initial attempts
    to realize the clustering tool employed the Gallagher-Humblet-Spira (GHS) distributed
    minimum spanning tree algorithm\autocite[102-106]{DNA}. Development of a $k$-machine
    model\autocite[129-133]{DNA} using Tcl threads as the individual machines was started but ultimately
    abandoned due to low development velocity. In its place, a simple breadth-first
    flooding approach\autocite[55-58]{DNA} was adopted and found to be sufficient to the task.

    The call records themselves are stored in a format that is searchable and aggregable
    via standard SQL statements. This provides the developer with an expressive foundation
    for data analysis and visualization which has long enjoyed wide adoption with a
    minimal learning curve.
  }
}

\section{Implementation}{
  As a pure Tcl implementation, the TURTLES project relies heavily on the
  introspection facilities in the language, particularly trace handlers.
  This section examines some of the implementation details and provides
  justification for the decisions made.

  \subsection{Integration}{
    For most users, installing the library in the Tcl search path and placing
    the directive \texttt{package require turtles} inside the target program's source
    should suffice to enable usage of the library. The project \texttt{README.md} contains
    instructions on installation, as well as commencement and termination of data
    collection in the ostensibly common case. The particulars of usage and data collection
    are covered in following subsections.
  }

  \subsection{Parameterization}{
    A set of command-line arguments are used to parameterize the call record colletion
    so that the user may specify parameters at runtime without having to modify a program's source.
    The command-line parameters themselves are bracketed by the literals \texttt{+TURTLES} and \texttt{-TURTLES}
    to distiguish arguments meant for the TURTLES library from arguments to the program.
    The arguments to TURTLES may interspersed anywhere in the command line provided that
    they are appropriately bracketed.

    During initialization, \texttt{::turtles::options::consume} proc the name reference of an
    \texttt{argv}-style string and destructively consumes the TURTLES parameters from that variable.
    Therefore, it is  strongly recommended to place the initialization call to
    \texttt{::turtles::release\_the\_turtles} as close to the main entry point of the target program
    as possible, preferably before the program's own command-line argument parsing.

    Extensive instructions on command-line usage is available in the doxygen code documentation
    for \texttt{turtles.tcl}, but the gist of each option is replicated here for the convenience of the reader.

    \subsubsection{\texttt{-enabled}}{
      The TURTLES library remains dormant unless explicitly enabled. This enables programs to
      be instrumented with the TURTLES library without having to require boolean debug flags
      in the source. If the flag is not set at the command-line, collection will not happen
      since the trace handlers and persistence mechanisms will not be created at all, avoiding
      even idle overhead.
    }
    \subsubsection{\texttt{-commitMode (direct|staged) [-intervalMillis <ms>] [-scheduleMode (ev|mt)]}}{
      As mentioned in the Design section, the TURTLES library can persist call records
      either directly to final storage or indirectly to ephemeral storage which is
      persisted to final storage at regular intervals. For the indirect staged mode,
      the interval is specified in milliseconds as an argument to the \texttt{-intervalMillis}
      command-line switch. For direct mode, the \texttt{-intervalMillis} argument may be
      omitted. The default mode is \texttt{staged}, and the default finalization
      interval is \texttt{30000}, or 30s.

      Persistence is implemented through the use of SQLite databases. Ephemeral
      storage is an in-memory database, while final storage is a database file
      that is attached to the in-memory database in staged mode and immediately
      updated in direct mode.

      The \texttt{-scheduleMode} option indicates the method of dispath for the periodic
      finalization in staged commit mode. Under multithreaded mode, specified by \texttt{mt},
      the TURTLES library launches two separate Tcl threads, i.e., a recorder and a scheduler.
      The recorder handles all call record storage updates sequentially according to its thread
      lambda message queue including finalization operations. The scheduler periodically sends
      finalizing lambdas to be processed by the recorder according to the configured interval.
      Under event-loop mode, specified by \texttt{ev}, the TURTLES library launchs a self-recurring
      after job into the main Tcl event loop. This mode assumes that the event loop is entered
      subsequently after TURTLES initialization.
    }
    \subsubsection{\texttt{-dbPath <directory> -dbPrefix <string>}}{
      The \texttt{-dbPath} option specifies the directory path only for the SQLite database serving
      as final storage. The actual database filename is specified by convention
      informed by the \texttt{-dbPrefix} option. The full filepath is
      determined as \texttt{<dbPath>/<dbPrefix>-<pid>.db} where \texttt{<pid>}
      is the operating system process ID of the program. By default, this will
      yield the filename \texttt{./turtles-<pid>.db}, i.e., \texttt{turtles-<pid>.db}
      in the current working directory during program execution. The path is canonicalized
      to an absolute path by the helper \texttt{::turtles::persistence::base::get\_db\_filename}.
    }
    \subsubsection{\texttt{-debug}}{
      Including the \texttt{-debug} flag in the TURTLES options turns on verbose logging
      of call record collection to \texttt{stderr}. This is useful during development to
      ensure calls are appropriately recorded but is not recommended for production runs.
    }
  }

  \subsection{Initialization}{
    Commencement of record collection is effected through calling \texttt{::turtles::release\_the\_turtles}
    with the \texttt{argv}-style command-line argument string reference.

    \subsubsection{Call Record Persistence}{
      Based on the parameters provided by the user, the TURTLES library opens the SQLite databases that
      serve as the requisite final and ephemeral call record storage.
    }
    \subsubsection{Trace Handlers}{
      A bootstrapping trace handler is attached to the \texttt{execution leave} event of the \texttt{proc} command.
      This handler has to determine the fully-qualified original name of the defined proc and
      add a handler each to the defined proc's \texttt{execution enter} and \texttt{execution leave} events,
      so it must be done post-definition so the two handlers can attach without an error.

      Additionally, the Tclx package is loaded and its version of the \texttt{fork} command is instrumented
      with handlers on \texttt{execution enter} and \texttt{execution leave}. This permits for
      some mitigation in case of process forking, which was a requirement of the MMHF program that
      engendered this experiment.
    }
  }

  \subsection{Collection}{
    \subsubsection{Deterministic Identification}{
      \paragraph{Proc ID}{
        The original name of each proc was hashed to an integral value using a Rabin-Karp rolling hash\autocite{univhash}.
        The mapping outputs in the range $[0, M_8]$, where $M_8$ is the 8th Mersenne prime, or $2^{31}-1$.
        Every caller and callee proc defined by the \texttt{proc} command has an entry created in the proc
        table of the output DB.
      }
      \paragraph{Trace ID}{
        Because the enter and leave handlers are defined separately, there needs to be a way to communicate
        between the handlers to ensure that the entry and exit of a specific call point are properly associated.
        This was the impetus for adding the trace id to the call point record. In the first implementation,
        each trace ID was an integer list hash of the following values: the caller ID, the proc ID, and the
        proc entry microsecond timestamp. Initially, a stack was used whereby the enter handler would push
        onto the stack and the leave handler would pop it off under the assumption of single-threaded execution.

        However, this did not account for yielding and other vagaries of the Tcl event loop, and it was
        observed that the trace IDs would not match up frequently in testing with non-trivial programs.
        To protect against this, a deterministic scheme was used so that the trace ID was computed
        as the integer list hash of the thread ID, stack level, caller ID, source line, and called ID.
        This had the additional benefit of saving the memory required for the stack at the cost of extra compute time.
        The stack level and source line were included in consideration of cases where a caller calls the callee
        numerous times in its body. The multi-threaded case has not been tested since it requires an
        initialization of TURTLES in each thread, and it is expected that there will be bugs in the implementation
        under those conditions, especially with respect to contention for the call record final storage.
      }
    }

    \subsubsection{Enter Handler}{
      The proc enter handler records its own computation start time, computes the requisite hashes for the call record,
      records a proc start time, and dispatches a lambda containing the provisional record to the persistence mechanism.
      The only information missing is the call point completion time, which is supplied by the leave handler.
      The proc enter handler records its computation stop time and dispatches a record of its own work out of band
      so as to prevent infinite recursion on proc entry.
    }

    \subsubsection{Leave Handler}{
      The leave handler first captures the proc computation end time. It then computes the requisite hashes and
      dispatches a lambda containing the partial record to the persistence mechanism. Like the proc enter handler,
      it keeps its own record of computation time that it sends out of band.
    }
    \subsubsection{Call Record Persistence}{
      Under the multi-threaded scheduling mode, the recorder thread receives lambda messages to execute.
      Call records messages come from the enter and leave handlers associated with the defined procs
      whether for the procs under observation or the handlers' own out-of-band execution metrics.
      For the staged case, a scheduler thread periodically sends lambdas instructing the recorder to
      flush unfinalized call records from ephemeral to final storage.

      Under the event-loop scheduling mode, the scheduler is implemented as a self-recurring proc
      availing itself of Tcl's \texttt{after} mechanism. This requires the main program to enter the
      event loop at some point in time. Call records are immediately written to ephemeral storage
      in staged commit and final storage in direct commit mode.

      \paragraph{Fork Safety}{
        The MMHF program operates by performing some initialization and then forking multiple children of itself
        across which the stream processing load is balanced. Most of the procs to be instrumented are defined
        prior to the fork. In order to enable the TURTLES library to cross the fork boundary, the aforementioned
        enter handler on \texttt{Tclx::fork} finalizes whatever call records have been incurred during execution,
        shuts down the recorder and scheduler threads or event loop jobs and closes the fork parent's final storage.

        The trace handlers as occupants of memory remain after the fork, and so the leave handler on \texttt{Tclx::fork}
        determines the fork child's process ID as well as the parent's for use in reconstructing the call record storage.
        The parent SQLite database is copied to a new child database, and since both databases are deterministically
        named by a convention that includes the process ID, conflicts should be minimized. Both children and parent
        reload their databases post-fork and restart the recorder and scheduler workers as applicable.

        There is a small window for lacunae in the call record history at the time of fork, but this is deemed
        an acceptable loss in light of the benefit that the implicit fork safety provides. Every child should have
        roughly the same information as the parent and its siblings pre-fork, subject to spawn order. This adheres
        to the copy semantics of forking at least by reasonable approximation if not exactly.

        Note that for long running programs, there is a potential for conflict if the parent process periodically
        refreshes the children and the operating system crosses the max PID boundary. In this case, a child may open
        a database that previously belonged to another child from the past. This is not a problem so long as their
        is a parent-child or sibling relationship between the processes that share the PID. The uniqueness of the
        trace ID for each call record is determined by both time (epoch execution microsecond stamps) and space
        (thread ID, call stack) should make post-hoc merging of all the associated SQLite databases trivial
        provided that the merge ignores conflicts, which most likely will arise from pre-fork records and the
        negligible probability of hash collisions in most use cases. Making multiple simultaneous runs of separate
        programs instrumented by TURTLES without specifying a unique database filename prefix per program
        will yield the expected and completely avoidable results.
      }
      \paragraph{Thread Safety}{
        Because the Tcl threading model defines a separate Tcl interpreter per thread, capturing call records
        in a multi-threaded scenario requires initialization at the start of each participating thread.
        This is not presently recommended since the behavior is untested, and as was previously alluded, it will
        probably result in contention for the call record database and possibly deadlock.

        The snippet message model used here incurs a substantial amount of overhead and requires that call record information
        be fully evaluated to literals since the threads do not share state. Thread state variables (TSVs) were
        eschewed to avoid the complexity of locking and state maintenance in favor of a stateless message-passing scheme.
      }
    }
  }

  \subsection{Finalization}{
    Finalization of call record storage is a simple process. Since SQLite permits attaching extraneous databases
    for information transfer, the finalization is done as a bulk insert query from ephemeral to final storage.
    Upon entry into the finalization procedure, the current clock is memoized and all records with non-null exit
    timestamps prior to the memoized cutoff are moved into final storage and deleted from ephemeral storage.
    Since the ephemeral storage has no need to maintain completed call records for update, this keeps the ephemeral
    storage size from growing without bound and improves the speed of update queries since there is fewer records
    to search.

    When the finalization procedure \texttt{::turtles::capture\_the\_turtles} is invoked,
    the applicable recorder and scheduler workers are stopped, and a final query moving call records
    from ephemeral storage to final storage is executed, including all records regardless of whether they had
    finished execution or not by the time the collection was ended.
  }
  }
}

\section{Experiments}{
  TBD
}

\section{Conclusions}{
  TBD
}

\section{Future Work}{
  TBD
}

\printbibliography

\end{document}
